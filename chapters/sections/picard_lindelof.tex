% !TeX root = ../../script.tex
\documentclass[../../script.tex]{subfiles}

\begin{document}
\section{The Picard-Lindelöf Theorem}

\begin{eg}
    Consider the ODE 
    \[
        y' = 2\sqrt{\abs{y}}
    \]
    Possible solutions are 
    \begin{align*}
        y(x) &= (x - c)^2 ~~c > 0 \\
        y(x) &= -(x - c)^2 ~~c < 0 \\
        y(x) = 0
    \end{align*}
    Another solution could be 
    \[
        y(x) = \begin{cases}
            -(x - a)^2, & x \in (-\infty, a) \\
            0, & x \in [a, b] \\
            (x - b)^2, & x \in (b, \infty)
        \end{cases}
    \]
    for $a, b \in \realn$ with $a \le b$ So the IVP $y(0) = 0$ has many solutions.
\end{eg}

\begin{defi}
    Let $D \subset \realn \times \realn^n$ be open, $(x_0, y_0) \in D$ and $f: D \rightarrow \realn^n$.
    We say $f$ fulfils a local Lipschitz-condition in the point $(x_0, y_0)$ if there exists a neighbourhood $U$ of $(x_0, y_0)$ such that 
    \[
        \norm{f(x, y) - f(x, z)} \le L \norm{y - z} ~~\forall (x, y), (x, z) \in U
    \]
\end{defi}

\begin{eg}
    Consider 
    \begin{align*}
        f: \realn &\longrightarrow \realn \\
        (x, y) &\longmapsto x^2y^2
    \end{align*}
    Then 
    \begin{align*}
        \abs{f(x, y) - f(x, z)} = \abs{x^2(y^2 - z^2)} &= \abs{x^2(y - z)(y + z)} \\
        &= \underbrace{\abs{x^2 (y + z)}}_{\alpha(x, y, z)} \abs{y - z}
    \end{align*}
    The function $\alpha(x, y, z)$ is unbounded, so the global Lipschitz condition isn't satisfied.
    Now choose a fixed $(x_0, y_0) \in \realn \times \realn$, and set 
    \[
        R > \max\set{\abs{x_0}, \abs{y_0}}
    \]
    Then $\forall (x, y), (x, z) \in (-R, R) \times (-R, R)$
    \[
        \alpha(x, y, z) \le R^2 \abs{y + z} \le R^2\left(\abs{y} + \abs{z}\right) \le 2R^3
    \]
    So $f$ fulfils a local Lipschitz condition in $(x_0, y_0)$.
\end{eg}

\begin{defi}
    Let $\measure$ be a measure space, $f: \Omega \rightarrow \realn^n$ measurable and $f_1, \cdots, f_n$ are the component functions of $f$. So 
    \[
        f(\omega) = \left(f_1(\omega), f_2(\omega), \cdots, f_n(\omega)\right)
    \]
    $f$ is said to be integrable if $f_1, \cdots, f_n$ are integrable, and we define
    \[
        \int f \dd\mu = \left(\int f_1 \dd\mu, \int f_2 \dd\mu, \cdots, \int f_n \dd\mu \right)
    \]
\end{defi}

\begin{thm}
    Let $\measure$ be a measure space, define $\dnorm$ to be the norm on $\realn^n$ and let $f: \Omega \rightarrow \realn^n$ be measurable.
    Then $f$ is integrable if and only if $\norm{f}$ is integrable, and 
    \[
        \norm{\int f \dd\mu} \le \int \norm{f} \dd\mu
    \]
\end{thm}
\begin{proof}
    Without proof.
\end{proof}

\begin{lem}\label{lem:picardlem}
    Let $D \subset \realn \times \realn^n$, $(x_0, y_0) \in D$ and $f: D \rightarrow \realn^n$ continuous.
    Let $I$ be an open interval and $y: I \rightarrow \realn^n$ be continuously differentiable, such that 
    $(x, y(x)) \in D ~~\forall x \in I$. Then $y$ is a solution of the IVP
    \begin{align*}
        y' = f(x, y) && y(x_0) = y_0
    \end{align*}
    if and only if $y$ satisfies the integral equation 
    \[
        y(x) = y_0 + \int_{x_0}^x f(t, y(t)) \dd{t}
    \]
\end{lem}
\begin{proof}
    Let $y$ fulfil the IVP. Then 
    \[
        y(x) - y_0 = y(x) - y(x_0) = \int_{x_0}^x y'(t) \dd{t} = \int_{x_0}^x f(t, y(t)) \dd{t}
    \]
    If $y$ fulfils the integral equation, then 
    \[
        y'(x) = f(x, y(x))
    \]
\end{proof}

\begin{thm}[Picard-Lindelöf Theorem]
    Let $D \subset \realn \times \realn^n$ be open, $(x_0, y_0) \in D$ and $f: D \rightarrow \realn^n$ continuous such that $f$ fulfils a local Lipschitz condition in $y$.
    Then $\exists \epsilon > 0$ such that the IVP 
    \begin{align*}
        y' = f(x, y) && y(x_0) = y_0
    \end{align*}
    has exactly one solution on $(x_0 - \epsilon, x_0 + \epsilon)$.
\end{thm}
\begin{proof}
    Let $U \subset D$ be a neighbourhood of $(x_0, y_0)$, such that 
    \begin{equation}
        \norm{f(x, y) - f(x, z)} \le L \norm{y - z} ~~\forall (x, y), (x, z) \in U
    \end{equation}
    Choose $a, r > 0$ such that 
    \begin{equation}
        \tilde{D} = [x_0 - a, x_0 + a] \times K_r(y_0) \subset U
    \end{equation}

    \begin{center}
        \begin{tikzpicture}
            \draw[thick, ->] (3, 0) -- (4, 0) node[right] {$x$};
            \draw[thick, ->] (3, 0) -- (3, 1) node[above] {$y$}; 

            \draw[thick] (0, 0) ellipse (2.5cm and 4cm);
            \node at (1.3, 3) {$D$};

            \draw[dashed] (2, -1.5) -- (2, 1.5) -- (-2, 1.5) -- (-2, -1.5) -- (2, -1.5);
            \node[below left] at (2, 1.5) {$\tilde{D}$};
            \draw[fill] (0, 0) circle[radius=1pt] node[below right] {$(x_0, y_0)$};

            \node[below] at (0, -1.5) {$a$};
            \node[left] at (-2, 0) {$r$};
        \end{tikzpicture}
    \end{center}
    $\tilde{D}$ is compact and $f$ is continuous, i.e. $f$ is bounded on $\tilde{D}$ by $M \in (0, \infty)$.
    \begin{equation}
        \norm{f(x, y)} \le M ~~\forall (x, y) \in \tilde{D}
    \end{equation}
    Choose an $\epsilon$ such that $0 < \epsilon \le a$ and such that 
    \begin{align}
        \epsilon M < r && \epsilon L < 1
    \end{align}
    Set $I := (x_0 - \epsilon, x_0 + \epsilon)$, and 
    \begin{equation}
        X = \set[y \text{ continuous}]{y: I \rightarrow K_r(y_0)} \subset C(I)
    \end{equation}
    $X$ is closed, and thus complete.
    Define $T: X \rightarrow X$ with 
    \begin{equation}
        T(y)(x) := y_0 + \int_{x_0}^x f(t, y(t)) \dd{t}
    \end{equation}
    We want to show $T(y) \subset X$:
    \begin{equation}
        \begin{split}
            \norm{T(y)(x) - y_0} = \norm{\int_{x_0}^x f(t, y(t)) \dd{t}} &\le \int_{x_0}^x \norm{f(t, y(t))} \dd{t} \\
            &\le M \int_{x_0}^x \dd{t} < \epsilon M < r
        \end{split}
    \end{equation}
    Now consider 
    \begin{equation}
        \begin{split}
            \norm{T(y)(x) - T(\tilde{y})(x)} &= \norm{\int_{x_0}^x(f(t, y(t)) - f(t, \tilde{y}(t))) \dd{t}} \\
            &\le \int_{x_0}^x \norm{f(t, y(t)) - f(t, \tilde{y}(t))} \dd{t} \\
            &\le \int_{x_0}^x L \cdot \norm{y(t) - \tilde{y}(t)} \dd{t} \\
            &\le \int_{x_0}^x L \supnorm{y - \tilde{y}} \le L\supnorm{y - \tilde{y}} \cdot \epsilon
        \end{split}
    \end{equation}
    By taking the supremum over all $x \in I$ we get 
    \begin{equation}
        \supnorm{T(y) - T(\tilde{y})} \le \underbrace{\epsilon L}_{< 1} \supnorm{y - \tilde{y}}
    \end{equation}
    So $T: X \rightarrow X$ is strictly contractive. 
    According to the Banach fixed point theorem, there exists a unique fixed point of $T$ in $x$, that means
    $\exists y \in X$ such that 
    \begin{equation}
        y_0 + \int_{x_0}^x f(t, y(t)) \dd{t} = T(y)(x) = y(x) ~~\forall x \in I
    \end{equation}
    Due to \Cref{lem:picardlem}, there eixsts a unique solution to the ODE.
\end{proof}

\begin{rem}
    One can approximate a fixed point by repeatedly applying $T$. For example consider
    \[
        \phi(x) = y_0
    \]
    and define 
    \begin{align*}
        \phi_0 = \phi && \phi_i = T(\phi_{i-1}) = y_0 + \int_{x_0}^x f(t, \phi_{i-1}(t)) \dd{t}
    \end{align*}
    This process is called Picard iteration, and the $\phi_i$ converge uniformly to the solution.
\end{rem}

\begin{eg}
    Consider 
    \[
        y' = \sqrt{\norm{y}}
    \]
    Then 
    \[
        \limes{y}{0} \left(\frac{\abs{f(x, y) - f(x, 0)}}{\abs{y - 0}}\right) = \limes{y}{0} \rec{\sqrt{\abs{y}}} \conv{} \infty
    \]
    Which means the local Lipschitz condition is not satisfied.
\end{eg}

\begin{thm}
    Let $D \subset \realn \times \realn^n$ be open, $f: D \rightarrow \realn^n$ continuously differentiable. 
    Then $f$ satisfies a local Lipschitz condition in terms of $y$.
\end{thm}
\begin{proof}
    Let $(x_0, y_0) \in D$. Choose $r > 0$ such that $\cball[r](x_0, y_0) \subset D$. 
    The total derivative $D_y f$ is continuous and thus bounded on $\cball[r](x_0, y_0)$.
    \begin{equation}
        \exists L > 0: ~~\norm{D_y f(x, y)} \le L ~~\forall (x, y) \in \cball[r](x_0, y_0)
    \end{equation}
    Applying the generalized mean value theorem yields 
    \begin{equation}
        \begin{split}
            \norm{f(x, y) - f(x, z)} &\le \sup_{t \in [0, 1]} \norm{D_y f(x, y + t(z - y))} \norm{y - z} \\
            &\le L \norm{y - z}
        \end{split}
    \end{equation}
    If $n=1$ we can specify 
    \begin{equation}
        \abs{f(x, y) - f(x, z)} = \abs{\partial_y f(x, \xi) (y - z)}
    \end{equation}
\end{proof}

\begin{eg}
    Consider 
    \[
        y'' = -\frac{y}{\norm{y}^3}
    \]
    The function 
    \begin{align*}
        f: \underbrace{\realn \times \realn^3 \setminus \set{0} \times \realn^3}_D &\longrightarrow \realn^3 \times \realn^3 \\
        (x, y, z) &\longmapsto \left(z, (y_1^2 + y_2^2 + y_3^2)^{-\frac{3}{2}} \cdot y \right)
    \end{align*}
    is continuously differentiable. So the IVP for arbitrary initial points in $D$ has a locally unique solution.
\end{eg}

\begin{defi}
    Let $D \subset \realn \times \realn^n$ be open, $(x_0, y_0) \in D$. A solution $\tilde{y}: \tilde{I} \rightarrow \realn^n$ of the IVP 
    \begin{align*}
        y' = f(x, y) && y(x_0) = y_0
    \end{align*}
    is said to be a (real) continuation of the solution $y: I \rightarrow \realn^n$ if $I \subset \tilde{I}$ and $y(x) = \tilde{y}(x) ~~\forall x \in I$.
    A solution $y$ is said to be a maximal solution if there are no real continuations.
\end{defi}

\begin{thm}
    Let $D \subset \realn \times \realn^n$ be open, $(x_0, y_0) \in D$ and $f: D \rightarrow \realn^n$ continuous and satisfying a local Lipschitz condition in terms of $y$.
    Then the IVP 
    \begin{align*}
        y' = f(x, y) && y(x_0) = y_0
    \end{align*}
    has a unique solution.
\end{thm}
\begin{proof}
    First, let $y: I \rightarrow \realn^n$ and $\tilde{y}: \tilde{I} \rightarrow \realn^n$ be solutions  of the IVP.
    Then $y = \tilde{y}$ on $I \cap \tilde{I} =: (a, b)$. Let 
    \begin{equation}
        c = \sup\set[{y = \tilde{y} \text{ on } [x_0, \tilde{c}]}]{\tilde{c} \in [x_0, b)}
    \end{equation}
    According to Picard-Lindelöf, such $\tilde{c}$ exist. Then there exists a sequence $\seq{c} \subset (x_0, c)$ such that $y = \tilde{y}$ on
    $[x_0, c_n) ~~\forall n \in \natn$ and $c_n \rightarrow c$.
    If $c < b$, then 
    \begin{equation}
        y(c) = \tilde{y}(c)
    \end{equation}
    because $y(c_n) = \tilde{y}(c_n) ~~\forall n \in \natn$. 
    The IVP
    \begin{align}
        u' = f(x, u) && u(c) = y(c)
    \end{align}
    has a locally unique solution on $(c - \epsilon, c + \epsilon) ~~\epsilon > 0$ according to Picard-Lindelöf.
    Since the $y$ and $\tilde{y}$ are both solutions to the IVP, they are identical on $(c - \epsilon, c + \epsilon)$.
    However, this contradicts the construction of $c$, so $c = b$.
    \begin{equation}
        \implies y = \tilde{y} \quad \text{on } [x_0, b)
    \end{equation}
    Analogously, one can prove $y = \tilde{y}$ on $(a, x_0]$.
    Now let $I_{\max}$ be the union of all open intervals that are domains of the solution of the IVP. 
    For $x \in I_{\max}$ we can choose 
    \begin{equation}
        y_{\max}(x) = y(x)
    \end{equation}
    for arbitrary solutions $y: I \rightarrow \realn$ with $x \in I$. So 
    \begin{equation}
        y_{\max}: I_{\max} \rightarrow \realn
    \end{equation}
    is a maximal solution that is unique.
\end{proof}

\begin{eg}
    \begin{enumerate}[(i)]
        \item Consider 
        \begin{align*}
            y' = e^{-y} && y(1) = 0
        \end{align*}
        The solution to this is 
        \begin{align*}
            y: (0, \infty] &\longrightarrow \realn \\
            x &\longmapsto \ln(x)
        \end{align*}
        and is maximal.

        \item Consider 
        \begin{align*}
            y' = -i \frac{y}{x^2} && y\left(\rec{2\pi}\right) = 1
        \end{align*}
        The solution to this is
        \begin{align*}
            (0, \infty) &\longrightarrow \cmpln \\
            x &\longmapsto e^{\frac{i}{x}}
        \end{align*}
        and is maximal.
    \end{enumerate}
\end{eg}

We define $\metric$ to be a metric space, $x \in X$ and $A \subset X$. Then 
\[
    d(x, A) = \inf\set[y \in A]{d(x, y)}
\]

\begin{thm}
    Let $D \subset \realn \times \realn^n$ be open, $(x_0, y_0) \subset D$ and $f: D \rightarrow \realn^n$ continuous and satisfying the local Lipschitz condition in terms of $y$.
    Let $a, b \in \realn \cup \set{-\infty, \infty}$ such that 
    \[
        -\infty \le a < x_0 < b \infty \infty
    \]
    and let 
    \[
        y: (a, b) \rightarrow \realn 
    \]
    a solution of the IVP 
    \begin{align*}
        y' = f(x, y) && y(x_0) = y_0
    \end{align*}
    Then $y$ is the maximal solution of the IVP if and only if one of these conditions 
    \begin{align*}
        \text{(i)} && b = \infty \\
        \text{(ii)} && \limes{x}{b} \norm{y(x)} = \infty \\
        \text{(iii)} && \limes{x}{b} d((x, y(x)), \boundary{D}) = 0
    \end{align*}
    and one of these 
    \begin{align*}
        \text{(i)} && a = -\infty \\
        \text{(ii)} && \limes{x}{a} \norm{y(x)} = \infty \\
        \text{(iii)} && \limes{x}{a} d((x, y(x)), \boundary{D}) = 0
    \end{align*}
    is fulfilled.
\end{thm}
\end{document}