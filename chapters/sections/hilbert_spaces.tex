% !TeX root = ../../script.tex
\documentclass[../../script.tex]{subfiles}

\begin{document}
    \section{Hilbert Spaces}

    \begin{defi}
        A mapping $\innerproduct{\cdot}{\cdot}: X \times X \longrightarrow \field$ with the properties
        \begin{enumerate}[(i)]
            \item $\innerproduct{x + y}{z} = \innerproduct{x}{z} + \innerproduct{y}{z}$
            \item $\innerproduct{\alpha x}{y} = \alpha \innerproduct{x}{y}$
            \item $\innerproduct{x}{y} = \conj{\innerproduct{y}{x}}$
            \item $\innerproduct{x} \ge 0, \quad \innerproduct{x} = 0 \iff x = 0$
        \end{enumerate}
        is called an inner product on $X$. A vector space $X$ with an inner product is said to be an inner product space.
    \end{defi}

    \begin{eg}
        Examples of inner product spaces are 
        \begin{enumerate}[(i)]
            \item Euclidean space $\realn^n$
            \[
                \innerproduct{x}{y} = \xi_1 \eta_1 + \cdots + \xi_n \eta_n
            \]

            \item Unitary space $\cmpln^n$
            \[
                \innerproduct{x}{y} = \xi_1 \conj{\eta_1} + \cdots + \xi_n \conj{\eta_n}
            \]

            \item Sequence space $l^2 = \set[\sum_{k=1}^{\infty} \abs{\xi_k}^2 < \infty]{x = \anyseqdef[\xi]{\field}}$
            \[
                \innerproduct{x}{y} = \sum_{k=1}^{\infty} \xi_k \conj{\eta_k}
            \]

            \item Space of square-integrable functions $L^2(A) = \set[\int_A \abs{f(t)}^2 \dd{t} < \infty]{f: A \rightarrow \field}$
            \[
                \innerproduct{x}{y} = \int_A f(t) \conj{g(t)} \dd{t}
            \]
        \end{enumerate}
    \end{eg}    

    \begin{defi}
        Define $\norm{x} = \sqrt{\innerproduct{x}}, ~x \in X$. This $\norm{\cdot}$ is a norm on $X$. A space $X$ with a norm induced by the inner product is called a normed space.
    \end{defi}

    \begin{lem}
        The Cauchy-Schwarz inequality holds
        \[
            \forall x, y \in X: \quad \abs{\innerproduct{x}{y}} \le \norm{x}\norm{y}
        \]
        as well as the triangle inequality 
        \[
            \forall x, y \in X: \quad \norm{x + y} \le \norm{x} + \norm{y}
        \]
    \end{lem}
    \begin{proof}
        \noproof
    \end{proof}

    \begin{rem}
        Consider the parallelogram equality 
        \[
            \norm{x + y}^2 + \norm{x - y}^2 = 2(\norm{x}^2 + \norm{y}^2)
        \]
        The norm $\norm{x} = \sqrt{\innerproduct{x}}$ satisfies this equality (without proof). This implies that $l^p, L^p(A)$ and $C(A)$ are not inner product spaces (for $p \ne 2$).
        This can be shown explicitly for $l^p$. Consider the sequences 
        \begin{align*}
            x &= (1, 1, 0, 0, \cdots) & y &= (1, -1, 0, 0, \cdots)
        \end{align*}
        Then $\norm{x} = \norm{y} = 2^{\frac{1}{p}}$ and $\norm{x + y} = \norm{x - y} = 2$.
        Thus the parallelogram equality doesn't hold 
        \[
            \norm{x + y}^2 + \norm{x - y}^2 = 2^2 + 2^2 \ne 2(2^{\frac{2}{p}} + 2^{\frac{2}{p}}) = 2(\norm{x}^2 + \norm{y}^2)
        \]
        unless $p = 2$.
    \end{rem}

    \begin{lem}
        Let $x_n \rightarrow x$ and $y_n \rightarrow y$ in $X$. Then $\innerproduct{x_n}{y_n} \rightarrow \innerproduct{x}{y}$.
    \end{lem}
    \begin{proof}
        \begin{equation}
            \begin{split}
                \abs{\innerproduct{x_n}{y_n} - \innerproduct{x}{y}} &= \abs{(\innerproduct{x_n}{y_n} - \innerproduct{x_n}{y}) + (\innerproduct{x_n}{y} - \innerproduct{x}{y})} \\
                &\le \abs{\innerproduct{x_n}{y_n} - \innerproduct{x_n}{y}} + \abs{\innerproduct{x_n}{y} - \innerproduct{x}{y}} \\
                &= \abs{\innerproduct{x_n}{y_n - y}} + \abs{\innerproduct{x_n - x}{y}} \\
                &\le \norm{x_n} \norm{y_n - y} + \norm{x_n - x}\norm{y} \conv{} 0
            \end{split}
        \end{equation}
    \end{proof}

    \begin{defi}
        An inner product space $X$ that is complete in the norm generated by the inner product is said to be a Hilbert space.

        A Hilbert space is a Banach space. A subspace $Y$ or an inner product space $X$ is defined to be a vector subspace  of $X$, with the inner product restricted to $Y \times Y$.
    \end{defi}

    \begin{thm}
        Let $Y$ be a subspace of a Hilbert space $H$. Then 
        \begin{enumerate}[(i)]
            \item $Y$ is complete $\iff$ $Y$ is closed in $H$ 
            \item $Y$ is finite-dimensional $\implies$ $Y$ is complete 
            \item $H$ is separable $\iff$ $Y$ is separable
        \end{enumerate}
        (A set $X$ is separable if $\exists M \subset X$ such that $M$ is dense in $X$)
    \end{thm}
    \begin{proof}
        \noproof
    \end{proof}

    \begin{defi}
        An element $x \in X$ is said to be orthogonal to an element $y \in X$ if $\innerproduct{x}{y} = 0$. One also says that $x$ and $y$ are orthogonal in that case, and it is denoted as $x \perp y$.
        Similarly, let $A, B \subset X$. Then
        \begin{align*}
            x \perp A &\iff \forall a \in A: \quad x \perp a \\
            A \perp B &\iff \forall a \in A ~\forall b \in B: \quad a \perp b
        \end{align*}
        Let $M$ be a non-empty subset of $X$, then the distance between $x$ and $M$ is defined as 
        \[
            \delta = \inf_{y \in M} \norm{x - y}
        \] 
        A subset $M \subset X$ is said to be convex if 
        \[
            \forall x, y \in M ~\forall \alpha \in [0, 1]: \quad (\alpha x + (1 - \alpha) y) \in M
        \]
    \end{defi}

    \begin{thm}\label{thm:17.11}
        Let $X$ be an inner product space and $M$ a non-empty, complete, convex subset of $X$. Then for every $x \in X$ there exists a unique $y \in M$ such that 
        \[
            \delta = \inf_{\tilde{y} \in M} \norm{x - \tilde{y}} = \norm{x - y}
        \]
    \end{thm}
    \begin{hproof}
        Consider a sequence $\anyseqdef[y]{M}$ such that $\delta_n = \norm{x - y_n} \conv{n \rightarrow \infty} \delta$.
        If we can show that this is a Cauchy sequence in $M$, then we can be sure that such a $y \in M$ exists and $y_n \conv{n \rightarrow \infty} y$.
    \end{hproof}

    \begin{cor}\label{cor:17.12}
        If $M = Y$, where $Y$ is a complete subspace of $X$ and $x \in X$ is fixed, then $z = x - y$ is orthogonal to $Y$.
    \end{cor}

    \begin{defi}
        Let $H$ be a Hilbert space and $Y$ a closed subspace of $H$. Then the set 
        \[
            Y^{\perp} = \set[z \perp Y]{z \in H}
        \]
        is the orthogonal complement of $Y$, which is a vector subspace of $H$.
    \end{defi}

    \begin{thm}
        Let $Y$ be a complete subspace of $X$. Then 
        \[ 
            \forall x \in X ~\exists! y \in Y, z \in Y^{\perp}: \quad x = y + z
        \]
    \end{thm}
    \begin{proof}
        The existence of $y$ and $z$ are ensured by \Cref{thm:17.11} and \Cref{cor:17.12}, if we choose a $y \in Y$ such that 
        \begin{equation}
            \inf_{\tilde{y} \in Y} \norm{x - \tilde{y}} = \norm{x - y}
        \end{equation}
        and $z = x - y$. Then $z \in Y^{\perp}$, so 
        \begin{equation}
            x = y + x - y = y + z
        \end{equation}
        To show that $y$ and $z$ are unique, assume that $x = y + z = y_1 + z_1$ with $y,y_1 \in Y$ and $z, z_1 \in Y^{\perp}$.
        Then $Y \ni y - y_1 = z_1 - z \in Y^{\perp}$ and 
        \begin{equation}
            \innerproduct{y - y_1}{z_1 - z} = \innerproduct{y - y_1}{y - y_1} = 0
        \end{equation}
        since $Y \perp Y^{\perp}$. This implies $y_1 = y$, and also $z_1 = z$.
    \end{proof}

    \begin{defi}
        A vector space $X$ is said to be adirect sum of two subspaces $Y$ and $Z$ of $X$, if 
        \[
            \forall x \in X ~\exists! y \in Y, z \in Z: \quad x = y + z
        \]
        It is notated as $X = Y \oplus Z$.
    \end{defi}

    \begin{rem}
        Let $Y$ be a closed subspace. Then $X = Y \oplus Y^{\perp}$.
    \end{rem}

    \begin{defi}
        An orthogonal set $M$ in $X$ is a subset of $X$ whose elements are pairwise orthogonal 
        \[
            \forall x, y \in M, ~x \ne y: \quad \innerproduct{x}{y} = 0
        \]
        An orthogonal set $M$ is said to be orthonormal if 
        \[
            \innerproduct{x}{y} = \begin{cases}
                1, & x = y \\
                0, & x \ne y
            \end{cases}
        \]
    \end{defi}

    \begin{eg}
        \begin{enumerate}[(i)]
            \item The sets 
            \begin{align*}
                M &= \set{(1, 0, 0), (0, 1, 0), (0, 0, 1)} \\
                M &= \set{\left(\rec{\sqrt{2}}, \rec{\sqrt{2}}, 0 \right), \left(\rec{\sqrt{2}}, -\rec{\sqrt{2}}, 0 \right), (0, 0, 1)}
            \end{align*}
            are orthonormal in $X = \realn^3$

            \item Let $X = l^2$. Then the set $M = \set[n > 0]{e_n}$ (with $e_1 = (1, 0, 0, \cdots), e_2 = (0, 1, 0, \cdots)$ and so on) is an orthonormal set
            
            \item Let $X = L^2([0, 2\pi])$. Then the sets $M = \set[n \ge 0]{e_n}$ with
            \[
                e_0(t) = \rec{\sqrt{2\pi}}, \quad e_n(t) = \frac{\cos nt}{\sqrt{\pi}}
            \]
            and $M = \set[n > 0]{e_n}$ with 
            \[
                e_n(t) = \frac{\sin nt}{\sqrt{\pi}}
            \]
            are orthonormal sets
        \end{enumerate}
    \end{eg}

    \begin{rem}
        Let $M = \set{e_1, \cdots, e_n}$ be a basis in $X$. Then 
        \[
            \forall x \in X ~\exists! \alpha_1, \cdots, \alpha_n: \quad x = \alpha_1 e_1 + \cdots + \alpha_n e_n
        \]
        If $M$ is orthonormal, i.e. $\innerproduct{e_k}{e_l} = \delta_{kl}$, then 
        \begin{align*}
            \innerproduct{x}{e_k} &= \innerproduct{\alpha_1 e_1 + \cdots + \alpha_k e_k + \cdots + \alpha_n e_n}{e_k} \\
            &= \alpha_1 \innerproduct{e_1}{e_k} + \cdots + \alpha_k \innerproduct{e_k}{e_k} + \cdots + \alpha_n \innerproduct{e_n}{e_k} = \alpha_k
        \end{align*}
    \end{rem}

    \begin{rem}
        The idea of the previous remark can be extended to infinite-dimensional inner product spaces. Let $\set{e_1, \cdots, e_n}$ be an orthonormal set in an infinite-dimensional space $X$.
        With some $x \in X$, consider 
        \[
            y := \sum_{k=1}^n \innerproduct{x}{e_k} e_k, \quad z := x - y
        \]
        By applying the Pythagorean theorem we get 
        \begin{align*}
            \innerproduct{z}{y} &= \innerproduct{x - y}{y} \innerproduct{x}{y} - \innerproduct{y}{y} = \innerproduct{x}{\sum_{k=1}^n \innerproduct{x}{e_k} e_k} - \norm{\sum_{k=1}^n \innerproduct{x}{e_k} e_k}^2 \\
            &= \sum_{k=1}^n \conj{\innerproduct{x}{e_k}} \innerproduct{x}{e_k} - \sum_{k=1}^n \norm{\innerproduct{x}{e_k} e_k}^2 = \sum_{k=1}^n \abs{\innerproduct{x}{e_k}}^2 - \sum_{k=1}^n \abs{\innerproduct{x}{e_k}}^2 \norm{e_k}^2 = 0
        \end{align*}
        Again, by using the Pythagorean theorem it appears that 
        \[
            \norm{x}^2 = \norm{y}^2 + \norm{z}^2 \ge \norm{y}^2 = \sum_{k=1}^n \abs{\innerproduct{x}{e_k}}^2
        \]
    \end{rem}

    \begin{thm}[Bessel Inequality]
        Let $\set[k > 0]{e_k}$ be an orthonormal sequence in an inner product space $X$. Then
        \[
            \forall x \in X: \quad \sum_{k=1}^{\infty} \abs{\innerproduct{x}{e_k}}^2 \le \norm{x}^2
        \]
    \end{thm}

    \begin{rem}
        Let $\set[n > 0]{x_n}$ be linearly independent. We want to construct a set $\set[n > 0]{e_n}$ with the property
        \[
            \forall n > 0: \quad \spn\set{x_1, \cdots, x_n} = \spn\set{e_1, \cdots, x_n}
        \]
        This can be achieved using the Gram-Schmidt procedure:
        \[
            e_1 := \frac{x_1}{\norm{x_1}}
        \]
        \[
            v_2 := x_2 - \innerproduct{x_2}{e_1} e_1, \quad e_2 := \frac{v_2}{\norm{v_2}}
        \]
        and in general 
        \[
            v_n := x_n - \sum_{k=1}^{n-1} \innerproduct{x_n}{e_k} e_k, \quad e_n := \frac{v_n}{\norm{v_n}}
        \]
    \end{rem}

    \begin{thm}
        Let $\set[k > 0]{e_k}$ be an orthonormal set in a Hilbert space $H$. Then the series 
        \[
            \sum_{k=1}^{\infty} \alpha_k e_k, \quad \alpha_k \in \field
        \]
        converges in $H$ if and only if 
        \[
            \sum_{k=1}^{\infty} \abs{\alpha_k}^2 < \infty
        \]
        If the initial sequence converges and 
        \[
            x := \sum_{k=1}^{\infty} \alpha_k e_k
        \]
        then $\alpha_k = \innerproduct{x}{e_k}$. For every $x \in H$ the series 
        \[
            \sum_{k=1}^{\infty} \innerproduct{x}{e_k} e_k
        \]
        converges, but not necessarily to $x$.
    \end{thm}
    \begin{proof}
        Proving that the sequence in question converges if and only if $\sum_{k=1}^{\infty} \abs{\alpha_k}^2$ converges is equivalent to proving that 
        $S_n = \alpha_1 e_1 + \cdots + \alpha_n e_n$ is Cauchy sequence if and only if $R_n = \abs{\alpha_1}^2 + \cdots + \abs{\alpha_n}^2$ is a Cauchy sequence.
        We can compute for $n < m$
        \begin{equation}
            \norm{S_m - S_n}^2 = \norm{\alpha_{n+1} e_{n+1} + \cdots + \alpha_m e_m}^2 = \abs{\alpha_{n+1}}^2 + \cdots + \abs{\alpha_m}^2 = R_m - R_n
        \end{equation}
        This does prove that $(S_n)$ is a Cauchy sequence in $H$ if and only if $(R_n)$ is a Cauchy sequence in $\realn$.

        Now we want to prove the second statement. For this, let $x \in \sum_{k=1}^{\infty} \alpha_k e_k$. We can compute for $k \le n$ that $\innerproduct{S_n}{e_k} = \alpha_k$.
        Since $S_n \conv{n \rightarrow \infty} x$ by the continuity of the inner product, it follows that 
        \begin{equation}
            \alpha_k = \innerproduct{S_n}{e_k} \conv{n \rightarrow \infty} \innerproduct{x}{e_k}
        \end{equation}

        The final statement follows from the Bessel inequality:
        \begin{equation}
            \sum_{k=1}^{\infty} \abs{\innerproduct{x}{e_k}}^2 \le \norm{x}^2 \implies \sum_{k=1}^{\infty} \abs{\innerproduct{x}{e_k}}^2 < \infty \implies \sum_{k=1}^{\infty} \innerproduct{x}{e_k} e_k < \infty
        \end{equation}
    \end{proof}

    \begin{defi}[Total Orthonormal Sets]
        A set $M \subset X$ is said to be a total orthonormal set, if $\closure{\spn{M}} = X$. Or in other words if $\spn{M}$ is dense in $X$.
        A total orthonormal family in $X$ is called an orthonormal basis.
    \end{defi}

    \begin{thm}
        In every Hilbert space $H$ there exists a total orthonormal set.
    \end{thm}
    \begin{proof}
        \noproof
    \end{proof}

    \begin{thm}[Parseval Equality]
        Let $M$ be an orthonormal set in a Hilbert space $H$. Then $M$ is total in $H$ if and only if 
        \[
            \forall x \in H: \quad \sum_k \abs{\innerproduct{x}{e_k}}^2 = \norm{x}^2
        \]
    \end{thm}
    \begin{proof}
        \noproof
    \end{proof}

    \begin{thm}
        Let $H$ be a Hilbert space. Then 
        \begin{enumerate}[(i)]
            \item If $H$ is separable, then every orthonormal set in $H$ is countable
            \item If $H$ contains a total orthonormal sequence, then $H$ is separable
        \end{enumerate}
    \end{thm}
\end{document}