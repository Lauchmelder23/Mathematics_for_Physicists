\documentclass[../../script.tex]{subfiles}

% !TEX root = ../../script.tex

\begin{document}
\section{Matrices and Gaussian elimination}
\begin{defi}
Let $a_{ij} \in \field$, with $i \in \set{1, \cdots, n}$, $j \in \set{1, \cdots, m}$. Then
\[
\begin{pmatrix}
	a_{11} & a_{12} & \cdots & a_{1m} \\
	a_{21} & a_{22} & \cdots & a_{2m} \\
	\vdots & \vdots & \ddots & \vdots \\
	a_{n1} & a_{n2} & \cdots & a_{nm}
\end{pmatrix}
\]
is called an $n \times m$-matrix. $(n, m)$ is said to be the dimension of the matrix. An alternative notation is
\[
	A = (a_{ij}) \in \field^{n \times m}
\]
$\field^{n\times m}$ is the space of all $n \times m$-matrices. The following operations are defined for $A, B \in \field^{n \times m}$, $C \in \field^{m \times l}$:
\begin{enumerate}[(i)]
	\item Addition
	\[
		A + B = 
		\begin{pmatrix}
			a_{11} + b_{11} & \cdots & a_{1m} + b_{1m} \\
			\vdots & \ddots & \vdots \\
			a_{n1} + b_{n1} & \cdots & a_{nm} + b_{nm}
		\end{pmatrix}
	\]
	
	\item Scalar multiplication
	\[
		\alpha \cdot A = 
		\begin{pmatrix}
			\alpha a_{11} & \cdots & \alpha a_{1m} \\
			\vdots & \ddots & \vdots \\
			\alpha a_{n1} & \cdots & \alpha a_{nm}
		\end{pmatrix}
	\]
	
	\item Matrix multiplication
	\[
		A \cdot C = 
		\begin{pmatrix}
			a_{11}c_{11}+a_{12}c_{21}+\cdots+a_{1m}c_{m1} & \cdots & a_{11}c_{1l}+a_{12}c_{2l}+\cdots+a_{1m}c_{ml} \\
			\vdots & \ddots & \vdots \\
			a_{n1}c_{11}+a_{n2}c_{21}+\cdots+a_{nm}c_{m1} & \cdots & a_{n1}c_{1l}+a_{n2}c_{2l}+\cdots+a_{nm}c_{ml}
		\end{pmatrix}
	\]
	or in shorthand notation
	\[
		(AC)_{ij} = \series[m]{k} a_{ik}c_{kj}
	\]
	
	\item Transposition
	
	The transposed matrix $A^T \in \field^{m \times n}$ is created by writing the rows of $A$ as the columns of $A^T$ (and vice versa).
	
	\item Conjugate transposition
	\[
		\conj{A} = \left(\overline{A}\right)^T
	\]
\end{enumerate}
\end{defi}

\begin{rem}\leavevmode
\begin{enumerate}[(i)]
	\item $\field^{n \times m}$ (for $n, m \in \natn$) is a vector space.

    \item $A \cdot B$ is only defined if  $A$ has as many columns as $B$ has rows.
    
    \item $\field^{n \times 1}$ and $\field^{1 \times n}$ can be trivially identified with $\field^n$.
    
    \item Let $A, B, C, D, E$ matrices of fitting dimensions and $\alpha \in \field$. Then
    \begin{align*}
        (A + B) C &= AC + BC \\
        A(B + C) &= AB + AC \\
        A(CE) &= (AC)E \\
        \alpha (AC) &= (\alpha A) C = A (\alpha C)
    \end{align*}
    \begin{align*}
        (A + B)^T &= A^T + B^T & \conj{(A + B)} &= \conj{A} + \conj{B} \\
        (\alpha A)^T &= \alpha (A)^T & \conj{(\alpha A)} &= \overline{A} \conj{A} \\
        (AC)^T &= C^T \cdot A^T & \conj{(AC)} &= \conj{C} \conj{A}
    \end{align*}
    \begin{proof}[Proof of associativity]
        Let $A \in \field^{n \times m}, C \in \field^{m \times l}, E \in \field^{l \times p}$. Furthermore let $i \in \set{1, \cdots, n}, j \in \set{1, \cdots, p}$.
        
		\begin{equation}
		\begin{split}
            \left((AC)E\right)_{ij} &= \sum_{k=1}^l (AC)_{ik} E_{kj} = \sum_{k=1}^l \left(\sum_{\tilde{k} = 1}^m a_{i\tilde{k}} c_{\tilde{k}k}\right) \cdot e_{kj} \\
			&= \sum_{k=1}^l \sum_{\tilde{k} = 1}^m a_{i\tilde{k}} \cdot c_{\tilde{k}k} \cdot e_{kj} \\
			&= \sum_{\tilde{k} = 1}^m a_{i\tilde{k}} \left( \sum_{k=1}^l c_{\tilde{k} k} e_{kj}\right) \\
			&= \sum_{\tilde{k} = 1}^m a_{i \tilde{k}} \cdot (CE)_{\tilde{k}j} \\
			&= (A(CE))_{ij}
        \end{split}	
		\end{equation}

		\begin{equation}
			\implies A(CE) = A(CE)
		\end{equation}
    \end{proof}

	\item Matrix multiplication is NOT commutative. First off, $AB$ and $BA$ are only well defined when $A \in \field^{n \times m}$ and $B \in \field^{m \times n}$. Example:
	\[
		\begin{pmatrix}
			0 & 1 \\ 0 & 0
		\end{pmatrix}
		\begin{pmatrix}
			0 & 0 \\ 1 & 0
		\end{pmatrix}
		=
		\begin{pmatrix}
			1 & 0 \\ 0 & 0
		\end{pmatrix}
		\neq
		\begin{pmatrix}
			0 & 0 \\ 1 & 0
		\end{pmatrix}
		\begin{pmatrix}
			0 & 1 \\ 0 & 0
		\end{pmatrix}
		=
		\begin{pmatrix}
			0 & 0 \\ 0 & 1
		\end{pmatrix}
	\]

	\item Let $n, m \in \natn$. There exists exactly one neutral additive element in $\field^{n \times m}$, which is the zero matrix. 
	Multiplication with the zero matrix yields a zero matrix.

	\item We define
	\[
		\delta_{ij} = \begin{cases}
				1 ,& i = j \\
				0 &\text{else}
		\end{cases}
	\]
	The respective matrix $I = (\delta_{ij}) \in \field^{n \times m}$ is called the identity matrix.

	\item $A \ne 0$ and $B \ne 0$ can still result in $AB = 0$:
	\[
		\begin{pmatrix}
			0 & 1 \\ 0 & 0
		\end{pmatrix}^2
		=
		\begin{pmatrix}
			0 & 0 \\ 0 & 0
		\end{pmatrix}
	\]
\end{enumerate}
\end{rem}

\begin{eg}[Linear equation system]
	Consider the following linear equation system
	\begin{align*}
		a_{11}x_1 + a_{12}x_2 + \cdots + a_{1m}x_m &= b_1 \\
		a_{21}x_1 + a_{22}x_2 + \cdots + a_{2m}x_m &= b_2 \\
		\vdots \\
		a_{n1}x_1 + a_{n2}x_2 + \cdots + a_{nm}x_m &= b_n \\
	\end{align*}
	This can be rewritten using matrices
	\begin{align*}
		A = \begin{pmatrix}
			a_{11} & \cdots & a_{1m} \\
			\vdots & \ddots & \vdots \\
			a_{n1} & \cdots & a_{nm}
		\end{pmatrix} 
		&&
		x = \begin{pmatrix}
			x_1 \\ \vdots \\ x_m
		\end{pmatrix}
		&&
		b = \begin{pmatrix}
			b_1 \\ \vdots \\ b_n
		\end{pmatrix}
	\end{align*}
	Which results in 
	\[
		Ax = B, ~~~A \in \field^{m \times n}, x \in \field^{m \times 1}, b \in \field^{n \times 1}	
	\]
	Such an equation system is called homogeneous if $b = 0$.
\end{eg}

\begin{thm}
Let $A \in \field^{n \times m}, b \in \field^n$. The solution set of the homogeneous equation system $Ax = 0$, (that means $\set[Ax = 0]{x \in \field^{m}} \subset \field^m$) is a linear subspace.
If $x$ and $\tilde{x}$ are solutions of the inhomogeneous system $Ax = b$, then $x-\tilde{x}$ solves the corresponding homogeneous problem.
\end{thm}
\begin{proof}
	$A \cdot 0 = 0$ shows that $Ax = 0$ has a solution. Let $x, y$ be solutions, i.e. $Ax = 0$ and $Ay = 0$. Then $\forall \alpha \in \field$:
	\begin{equation}
		A(x + \alpha y) = Ax + A(\alpha y) = \underbrace{Ax}_0 + \alpha(\underbrace{Ay}_0) = 0
	\end{equation}
	\begin{equation}
		\implies x + \alpha y \in \set[Ax = 0]{x \in \field^m}
	\end{equation}

	Next, let $x, \tilde{x}$ be solutions of $Ax = b$, i.e.
	\begin{equation}
		Ax = b, ~A\tilde{x} = b
	\end{equation}
	Then
	\begin{equation}
		A(x - \tilde{x}) = Ax - A\tilde{x} = b - b = 0
	\end{equation}
	Therefore, $x - \tilde{x}$ is the solution of the homogeneous equation system
\end{proof}

\begin{rem}[Finding all solutions]
	First find a basis $e_1, \cdots, e_k$ of 
	\[
		\set[Ax = 0]{x \in \field^m}
	\] 
	Next find some $x_0 \in \field^m$ such that $Ax_0 = b$. 
	Then every solution of $Ax = b$ can be written as
	\[
		x = x_0 + \alpha_1 e_1 + \cdots + \alpha_k e_k
	\]
\end{rem}

\begin{eg}
	Let 
	\begin{align*}
		A = \begin{pmatrix}
			1 & 2 & 0 & 0 & 1 \\
			0 & 0 & 1 & 0 & 0 \\
			0 & 0 & 0 & 1 & -1 \\
			0 & 0 & 0 & 0 & 0
		\end{pmatrix}
		&&
		b = \begin{pmatrix}
			1 \\ 2 \\ 3 \\ 4
		\end{pmatrix}
		&&
		c = \begin{pmatrix}
			3 \\ 2 \\ 1 \\ 0
		\end{pmatrix}
	\end{align*}
	Then $Ax = b$ has no solution, since the fourth row would state $0 = 4$. However, $Ax = c$ has the particular solution
	\[
		x = \begin{pmatrix}
			3 \\ 0 \\ 2 \\ 1 \\ 0
		\end{pmatrix}	
	\]
	If we consider the homogeneous problem $Ay = 0$, we can come up with the solution 
	\[
		y = \begin{pmatrix}
			-2 \\ 1 \\ 0 \\ 0 \\ 0
		\end{pmatrix}
		y_2 + 
		\begin{pmatrix}
			-1 \\ 0 \\ 0 \\ 1 \\ 1
		\end{pmatrix}
		y_5
	\]
	and in turn find the set of solutions
	\begin{align*}
		\set[Ay = 0]{y \in \field^5} &= \spn\set{(-2, 1, 0, 0, 0)^T, (-1, 0, 0, 1, 1)^T} \\
		\set[Ax = c]{x \in \field^5} &= \set{(3, 0, 2, 1, 0)^T + \alpha(-2, 1, 0, 0, 0)^T + \beta(-1, 0, 0, 1, 1)^T}
	\end{align*}
\end{eg}

\begin{defi}[Row Echelon Form]
	A zero row is a row in a matrix containing only zeros. The first element of a row that isn't zero is called the pivot.

	A matrix in row echelon form must meet the following conditions
	\begin{enumerate}[(i)]
		 \item Every zero row is at the bottom
		 \item The pivot of a row is always strictly to the right of the pivot of the row above it
	\end{enumerate}

	A matrix in reduced row echelon form must additionally meet the following conditions
	\begin{enumerate}[(i)]
		\item All pivots are $1$
		\item The pivot is the only non-zero element of its column
	\end{enumerate}
\end{defi}

\begin{rem}
	Let $A \in \field^{n \times m}$ and $b \in \field^n$. If $A$ is in reduced row echelon form, then $Ax = b$ can be solved through trivial rearranging.
\end{rem}

\begin{defi}[Matrix row operations]
Let $A$ be a matrix. Then the following are row operations
\begin{enumerate}[(i)]
	\item Swapping of rows $i$ and $j$
	\item Addition of row $i$ to row $j$
	\item Multiplication of a row by $\lambda \ne 0$
	\item Addition of row $i$ multiplied by $lambda$ to row $j$
\end{enumerate}
\end{defi}

\begin{thm}[Gaussian Elimination]
Every matrix can be converted into reduced row echelon form in finitely many row operations.
\end{thm}
\begin{proof}[Heuristic Proof]
	If $A$ is a zero matrix the proof is trivial. But if it isn't:
	\begin{itemize}
		\item Find the first column containing a non-zero element.
			\begin{itemize}
				\item Swap rows such that this element is in the first row
			\end{itemize}
		\item Multiply every other row with multiples of the first row, such that all other entries in that column disappear.
		\item Repeat, but ignore the first row this time
	\end{itemize}
	At the end of this the matrix will be in reduced row echelon form.
\end{proof}

\begin{defi}
	$A \in \field^{n \times n}$ is called invertible if there exists a multiplicative inverse. I.e.
	\[
		\exists B \in \field^{n \times n}: ~~AB = BA = I
	\]
	We denote the multiplicative inverse as $\inv{A}$
\end{defi}

\begin{rem}
	We have seen matrices $A \ne 0$ such that $A^2 = 0$. Such a matrix is not invertible.
\end{rem}

\begin{thm}
	Let $A, B, C \in \field^{n \times n}$, $B$ invertible and $A = BC$. Then
	\[
		A \text{ invertible} \iff C \text{ invertible}
	\]
	Especially, the product of invertible matrices is invertible.
\end{thm}
\begin{proof}
	Without proof.
\end{proof}

\begin{rem}
	Matrix multiplication with $A$ from the left doesn't "mix" the columns of matrix $B$
\end{rem}

\begin{thm}
	Let $A$ be a matrix, and let $\tilde{A}$ be the result of row operations applied to $A$. Then
	\[
		\exists T \text{ invertible}: ~~\tilde{A} = TA
	\]
	We say: The left multiplication with $T$ applies the row operations.
\end{thm}
\begin{proof}[Heuristic proof]
	You can find invertible matrices $T_1, \cdots, T_n$ that each apply one row operation. Then we can see that
	\begin{equation}
		\tilde{A} = \underbrace{T_n T_{n-1} \cdots T_1}_T A
	\end{equation}
	Since $T$ is the product of invertible matrices, it must itself be invertible.
\end{proof}

\begin{cor}
	Let $A \in \field^{n \times m}$, $b \in \field^n$, $T \in \field^{n \times m}$. Then $Ax = b$ and $TAx = Tb$ have the same solution sets.
\end{cor}
\begin{proof}
	If $Ax = b$ it is trivial that
	\begin{equation}
		Ax = b \implies TAx = Tb
	\end{equation}
	If $TAx = Tb$, then
	\begin{equation}
		Ax = \inv{T}TAx = \inv{T}Tb = b
	\end{equation}
\end{proof}

\begin{lem}
	Let $A \in field^{n \times m}$ be in row echelon form. Then
	\[
		A \text{ invertible} \iff \text{ The last row is not a zero row}	
	\]
	and
	\[
		A \text{ invertible} \iff \text{ All diagonal entries are non-zero}
	\]
\end{lem}
\begin{proof}
	Let $A$ be invertible with a zero-row as its last row. Then
	\begin{equation}
		(0, \cdots, 0, 1) \cdot A = (0, \cdots, 0, 0)
	\end{equation}
	Multiplying with $\inv{A}$ from the right would result in a contradiction. Therefore the last row of $A$ can't be a zero row.

	Now let the diagonal entries of $A$ be non-zero. This means we can use row operations to transform $A$ into the identity matrix, i.e.
	\begin{equation}
		\exists T \text{ invertible}: ~~TA = I \implies A = \inv{T}
	\end{equation}
\end{proof}

\begin{cor}
	Let $A \in \field^{n \times n}$. Then 
	\[
		A \text{ invertible} \iff \text{Every row echelon form has non-zero diagonal entries}	
	\]
	and
	\[
		A \text{ invertible} \iff \text{The reduced row echelon form is the identity matrix}	
	\]
\end{cor}
\begin{proof}
	Every row echelon form of $A$ has the form $TA$ with $T$ an invertible matrix. Especially, $\exists S \text{ invertible}$ such that $SA$ is 
	in reduced row echelon form. Then
	\begin{equation}
		TA \text{ invertible} \iff A \text{ invertible}
	\end{equation}
\end{proof}

\begin{rem}
	Let $A \in \field^{n \times n}$ be invertible, $B \in \field^{n \times m}$. Our goal is to compute $\inv{A}B$. First, write $(A \,\vert\, B)$.
	Now apply row operations until we reach the form $(I \,\vert\, \tilde{B})$. Let $S$ be the matrix realising these operations, i.e. $SA = I$. 
	Then $\tilde{B} = SB = \inv{A}B$. If $B = I$ this can be used to compute $\inv{A}$.
\end{rem}

\begin{eg}
	Let
	\[
		A = \begin{pmatrix}
			1 & 1 & 1 \\ 
			0 & 1 & 1 \\
			0 & 0 & 1
		\end{pmatrix}	
	\]
	Rewrite this as 
	\[
		\left(
		\begin{array}{@{}ccc|ccc@{}}
			1 & 1 & 1 & 1 & 0 & 0 \\
			0 & 1 & 1 & 0 & 1 & 0 \\
			0 & 0 & 1 & 0 & 0 & 1
		\end{array}
		\right)	
	\]
	Turn this into
	\[
		\left(
		\begin{array}{@{}ccc|ccc@{}}
			1 & 1 & 0 & 1 & 0 & -1 \\
			0 & 1 & 0 & 0 & 1 & -1 \\
			0 & 0 & 1 & 0 & 0 & 1
		\end{array}
		\right)	
	\]
	And finally
	\[
		\left(
		\begin{array}{@{}ccc|ccc@{}}
			1 & 0 & 0 & 1 & -1 & 0 \\
			0 & 1 & 0 & 0 & 1 & -1 \\
			0 & 0 & 1 & 0 & 0 & 1
		\end{array}
		\right)	
	\]
	The right part of the above matrix is $\inv{A}$.
\end{eg}

\begin{defi}
	Let $A \in \field^{n \times m}$ and let $z_1, \cdots, z_n \in \field^{1 \times m}$ be the rows of $A$. The row space of $A$ is defined as 
	\[
		\spn\set{z_1, \cdots, z_n}	
	\]
	The dimension of the row space is the row rank of the matrix. Analogously this works for the column space and the column rank.
	Later we will be able to show that row rank and column rank are always equal. They're therefore simply called rank of the matrix.
\end{defi}

\begin{thm}
	The row operations don't effect the row space.
\end{thm}
\begin{proof}
	It is obvious that multiplication with $\lambda$ and swapping of rows don't change the row space. Furthermore it is clear that every linear combination of
	$z_1 + z_2, z_2, \cdots, z_n$ is also a linear combination of $z_1, z_2, \cdots, z_n$, and vice versa.
\end{proof}

\begin{thm}
	Let $A$ be in row echelon form. Then the non-zero rows of the matrix are a basis of the row space of the matrix.
\end{thm}
\begin{proof}
	Let $z_1, \cdots, z_k \in \field^{1 \times n}$ be the non-zero rows of $A$. They create the space $\spn\set{z_1, \cdots, z_n}$, 
	since $z_k, \cdots z_n$ are only zero rows. Analogously,
	\begin{equation}
		\alpha_1 z_1 + \alpha_2 z_2 + \cdots + \alpha_k z_k = 0
	\end{equation}
	Let $j$ be the index of the column of the pivot of $z_1$. Then $z_2, \cdots, z_k$ have zero entries in the $j$-th column. Therefore
	\begin{equation}
		\alpha_1 \underbrace{z_{ij}}_{\ne 0} = 0 \implies \alpha_1 = 0
	\end{equation}
	By inductivity, this holds for every row.
\end{proof}

\begin{rem}
	\begin{enumerate}[(i)]
		\item To compute the rank of $A$, bring $A$ into row echelon form and count the non-zero rows.
		
		\item Let $v_1, \cdots, v_m \in \field^n$. To find a basis for 
		\[
			\spn\set{v_1, \cdots v_m}
		\]
		write $v_1, \cdots, v_m$ as rows of a matrix and bring it into row echelon form.
	\end{enumerate}
\end{rem}
\end{document}