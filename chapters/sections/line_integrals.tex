% !TeX root = ../../script.tex
\documentclass[../../script.tex]{subfiles}

\begin{document}
\section{Line Integrals}

\begin{defi}
    Let $I$ be an interval and $n \in \natn$. A parametrized curve (or path) in $\realn^n$ is a continuous mapping 
    \[
        \gamma: I \longrightarrow \realn^n
    \]
    A parametrized curve is said to be regular if it is $C^1$ and $\gamma'(t) \ne 0 ~~\forall t \in I$.
    It is said to be piecewise regular if there is a disjoint decomposition
    \[
        I = I_1 \cup I_2 \cup \cdots \cup I_n
    \]
    into partial intervals such that $\gamma$ is regular on each partial interval.

    A curve is a subset of $\realn^n$ that is the image of a parametrized curve. If $\curve$ is a curve, then 
    \[
        \gamma: I \longrightarrow \realn^n
    \]
    is said to be the parametrization of $\curve$, if $\gamma(I) = \curve$ and if $\gamma$ is injective on $\interior{I}$.
    The curves in this chapter will always be regular.
\end{defi}

\begin{eg}
    \begin{enumerate}[(i)]
        \item $\alpha, \kappa > 0$:
        \begin{align*}
            \gamma: \realn &\longrightarrow \realn^3 \\
            t &\longmapsto (\cos(\alpha t), \sin(\alpha t), \kappa t)
        \end{align*}
        This is the parametrization of a screw curve.

        \item The unit circle 
        \[
            \set[x^2 + y^2 = 1]{(x, y) \in \realn^2}
        \]
        is a curve with the parametrization
        \begin{align*}
            \gamma: [0, 2\pi] &\longrightarrow \realn^2 \\
            t &\longmapsto (\cos t, \sin t)
        \end{align*}

        \item A square 
        \[
            \set[\max\set{\abs{x_1}, \abs{x_2}} = 1]{(x, y) \in \realn^2}
        \]
        is a piecewise regular curve.
    \end{enumerate}
\end{eg}

\begin{rem}
    Let $\gamma: I \rightarrow \realn^n$ be regular, $f: \gamma(I) \rightarrow \realn$ be continuous and $a, b \in \interior{I}$.
    A decomposition $Z$ is given by the grid points
    \[
        a = t_0 < t_1 < \cdots < t_n = b
    \]
    The fineness of $Z$ is given by 
    \[
        m(Z) := \max_{t \in \set{0, 1, \cdots, n-1}} (t_{i+1} - t_i)
    \]
    We can represent $I$ in terms of $Z$ via 
    \[
        I(Z) := \sum_{i=0}^{n-1} f(\gamma(t_i)) \norm{\gamma(t_{i+1}) - \gamma(t_i)}
    \]
    Or in integral representation
    \[
        I(Z) = \int_a^b \underbrace{\sum_{i=0}^{n-1} f(\gamma(t_i)) \frac{\norm{\gamma(t_{i+1}) - \gamma(t_i)}}{\norm{t_{i+1} - t_i}} \charfun_{[t_i, t_{i+1})}(t)}_{g_Z(t)} \dd{t}
    \]
    So let $(Z_j)$ be a sequence of decompositions with 
    \[
        m(Z_j) \conv{j \rightarrow \infty} 0
    \]
    Let $t \in [a, b]$ not be a grid point of any $Z_j$. Then there exists a unique grid poiont $t_{j, i_j}$ such that $t \in [t_{j, i_j}, t_{j, i_{j+1}}]$. Then 
    \[
        \limes{j}{\infty} t_{j, i_j} = \limes{j}{\infty} t_{j, i_{j+1}} = t
    \]
    And thus 
    \[
        \limes{j}{\infty} g_{Z_j}(t) = f(\gamma(t)) \norm{\gamma'(t)}
    \]
    $\forall t$ that are not grid points of $Z_j$, this means tahat 
    \[
        g_{Z_j} \conv{j \rightarrow \infty} f \norm{\gamma'}
    \]
    almost everywhere. The dominated convergence theorem then tells us
    \[
        I(Z_j) = \int_a^b g_{Z_j}(t) \dd{t} \conv{j \rightarrow \infty} \int_a^b f(\gamma(t)) \norm{\gamma'(t)} \dd{t}
    \]
    Special case: For $f \equiv 1$ one gets the arc length.
\end{rem}

\begin{defi}[Line Integrals, Arc Length]
    Let $I$ be an interval and $\gamma: I \rightarrow \realn^n$ a parametrized curve. Define the functions
    \begin{align*}
        f: \gamma(I) \longrightarrow \realn && E: \gamma(I) \longrightarrow \realn^n
    \end{align*}
    Then 
    \[
        \int_{\gamma} f \dd{s} := \int_I f(\gamma(t)) \norm{\gamma'(t)} \dd{t}
    \]
    is said to be a scalar line integral (line integral of first kind), and 
    \[
        \int_{\gamma} \innerproduct{E}{\dd{s}} := \int_I \innerproduct{E(\gamma(t))}{\gamma'(t)} \dd{t}
    \]
    is said to be a vector line integral (line integral of second kind).
    The function $f$ or the vector field $E$ are integrable along $\gamma$ if the according integral exists.
    The integral 
    \[
        \int_{\gamma} \dd{s}
    \]
    is the arc length of $\gamma$, and $\gamma$ is said to be rectifiable if this integral is finite.

    If the curve $\gamma$ is closed, i.e. if $I = [a, b] ~~a, b \in \realn$ and 
    \[
        \gamma(a) = \gamma(b)
    \]
    Then the above integrals are often notated as 
    \begin{align*}
        \oint_{\gamma} \dd{s} && \oint_{\gamma} \innerproduct{E}{\dd{s}}
    \end{align*}
    to emphasize that the curve is closed. This changes nothing about the formulas, it is merely visual.
    I will try to adhere to this style.
\end{defi}

\begin{eg}[Circumference of the unit circle]
    Define 
    \begin{align*}
        \gamma: [0, 2\pi] &\longrightarrow \realn^2 \\
        t &\longrightarrow (\cos(t), \sin(t))
    \end{align*}
    and derive this function 
    \[
        \gamma'(t) = (-\sin(t), \cos(t)) \implies \norm{\gamma'(t)} = 1
    \]
    Then the circumference is 
    \[
        \oint_{\gamma} \dd{s} = \int_0^{2\pi} \dd{t} = 2\pi
    \]
\end{eg}

\begin{rem}
    \begin{enumerate}[(i)]
        \item If $\gamma$ is only piecewise regular then the integrands might not be defined for all $t$.
        \item Line integrals don't depend on the chosen parametrization. This means if $\curve$ is a curve and 
        \begin{align*}
            \gamma: I \rightarrow \curve && \rho: J \rightarrow \curve
        \end{align*}
        are parametrizations, then 
        \[
            \int_{\gamma} f \dd{s} = \int_{\rho} f \dd{s}
        \]
        We also write 
        \[
            \int_{\curve} f \dd{s}
        \]
        The same holds for vector integrals. 
        \item Both kinds of integrals depend on the scalar product.
        \item Both kinds of integrals are special cases of integrals over so called One-forms
    \end{enumerate}
\end{rem}

\begin{thm}
    Let $\gamma: I \rightarrow \realn^n$ be a parametrized curve, and $\vartheta: J \rightarrow I$ a diffeomorphism 
    (so $\vartheta \in C^1$ and $\vartheta'(t) \ne 0 ~~\forall t \in J$). Let $f: \gamma(I) \rightarrow \realn$, then 
    \[
        \int_{\gamma} f \dd{s} = \int_{\gamma \circ \vartheta} f \dd{s}
    \]
\end{thm}
\begin{proof}
    We can assume $I, J$ to be open, since the endpoints of the integrals are a null set and thus don't matter.
    W.l.o.g. let $\gamma$ be regular. Then 
    \begin{equation}
        \begin{split}
            \int_{\gamma \circ \vartheta} f \dd{s} &= \int_J f(\gamma \circ \vartheta)(t) \norm{(\gamma \circ \vartheta)'(t)} \dd{t} \\
            &= \int_J f(\gamma(\vartheta(t))) \norm{\gamma'(\vartheta(t)) \vartheta'(t)} \dd{t} \\
            &= \int_J f(\gamma(\vartheta(t))) \norm{\gamma'(\vartheta(t))} \abs{\vartheta'(t)} \dd{t} \\
            &= \int_I f(\gamma(\tau)) \norm{\gamma'(\tau)} \dd{\tau} \\
            &= \int_{\gamma} f \dd{s}
        \end{split}
    \end{equation}
\end{proof}

\begin{rem}
    \begin{enumerate}[(i)]
        \item One can show that for a curve $\curve$ and parametrizations
        \begin{align*}
            \gamma: I \rightarrow \curve && \rho: J \rightarrow \curve
        \end{align*}
        there exists a diffeomorphism $\vartheta: J \rightarrow I$ such that 
        \[
            \rho = \gamma \circ \vartheta
        \]
        So the line integral of first degree doesn't depend on the parametrization.

        \item A line integral of second degree doesn't depend on the parametrization if the parametrizations run along the curve in the same direction.
        So if $\vartheta' > 0$, $\vartheta$ is said to conserve orientation. If $\vartheta' < 0$ then the integral switches sign.
    \end{enumerate}
\end{rem}

\begin{eg}
    Let $\gamma: I \rightarrow \realn^3$ be the trajectory of a point mass, and $F: \realn^3 \rightarrow \realn^3$ a time-independent forcefield.
    The work done is then given by 
    \[
        W := \int_{\gamma} \innerproduct{F}{\dd{s}}
    \]
    The fact that the parametrization can be chosen arbitrarily means that the work done in a forcefield is independent from the velocity of the point mass.
\end{eg}

\begin{rem}
    \begin{enumerate}[(i)]
        \item Line integrals are linear in $f$ or $E$, meaning for 
        \[
            f, g: \gamma(I) \rightarrow \realn, ~~\lambda \in \realn
        \]
        we have
        \[
            \int_{\gamma}(g + \lambda g) \dd{s} = \int_{\gamma} f \dd{s} + \lambda \int_{\gamma} g \dd{s}
        \]

        \item Parametrized curves over compact intervals can be reparametrized so that $I = [0, 1]$.
        
        \item Let 
        \begin{align*}
            \gamma: [0, 1] \rightarrow \realn^n && \rho: [0, 1] \rightarrow \realn^n
        \end{align*}
        be curves with $\gamma(1) = \rho(0)$. Define 
        \begin{align*}
            \inv{\gamma}: [0, 1] &\longrightarrow \realn^n & \gamma\rho: [0, 1] &\longrightarrow \realn^n \\
            t &\longrightarrow \gamma(1 - t) & t &\longrightarrow \begin{cases}
                \gamma(2t), & t \le 0.5 \\
                \rho(2t + 1), & t > 0.5
            \end{cases}
        \end{align*}
        Then we have 
        \begin{align*}
            &\int_{\inv{\gamma}} f \dd{s} = \int_{\gamma} f \dd{s} \\
            &\int_{\gamma\rho} f \dd{s} = \int_{\gamma} f \dd{s} + \int_{\rho} f \dd{s} \\
            &\int_{\inv{\gamma}} \innerproduct{E}{\dd{s}} = -\int_{\gamma} \innerproduct{E}{\dd{s}} \\
            &\int_{\gamma\rho} \innerproduct{E}{\dd{s}} = \int_{\gamma} \innerproduct{E}{\dd{s}} + \int_{\rho} \innerproduct{E}{\dd{s}}
        \end{align*}
    \end{enumerate}
\end{rem}

\begin{defi}
    Let $U \subset \realn^n$ be open and $f: U \rightarrow \realn$ a $C^1$-function. Define 
    \[
        \grad f = (\partial_1 f, \partial_2 f, \cdots, \partial_m f)
    \]
    The vector field $E: U \rightarrow \realn^n$ is said to be conservative if there is a function $g: U \rightarrow \realn$ such that 
    \[
        E = \grad{g}
    \]
    $g$ is the potential of $E$.
\end{defi}

\begin{rem}
    \begin{enumerate}[(i)]
        \item In physics the sign is typically switched, so 
        \[
            E = -\grad{g}
        \]

        \item The IDE 
        \[
            p(x, y) + q(x, y)y' = 0
        \]
        is exact if and only if the vector field $(p, q)$ is conservative.

        \item If $E$ is conservative and $C^1$, then 
        \[
            \partial_i E_j = \partial_j E_i
        \]
        This condition is not sufficient in general.

        \item If $g$ is a potential for $E$, then the functions
        \[
            g + c ~~c \in \realn
        \]
        are also potentials.

        \item If $E$ is conservative, $g$ a potential and $\gamma: [a, b] \rightarrow \realn^n$ a curve, then
        \begin{align*}
            \int_{\gamma} \innerproduct{E}{\dd{s}} &= \int_a^b \innerproduct{E(\gamma(t))}{\gamma'(t)} \dd{t} \\
            &= \int_a^b \left(\partial_1 g(\gamma(t)) \gamma_1'(t) + \cdots + \partial_n g(\gamma(t))\gamma_n'(t)\right) \dd{t} \\
            &= \int_a^b (g \circ \gamma)'(t) \dd{t} = g(\gamma(b)) - g(\gamma(a))
        \end{align*}
        The vector line integral over conservative fields is independent from the chosen path (it only depends on the start and end points).

        \item Let $U$ be open, path-connected and $E: U \rightarrow \realn^n$ a conservative vector field. 
        Choose a fixed $x_0 \in U$, and for $x \in U$ choose a parametrized curve $\gamma_x$ from $x_0$ to $x$.
        Then 
        \[
            x \longmapsto \int_{\gamma_x} \innerproduct{E}{\dd{s}}
        \]
        is a potential, because if $g$ is an arbitrary potential we have 
        \[
            \int_{\gamma_x} \innerproduct{E}{\dd{s}} = g(x) - g(x_0) \quad \forall x \in U
        \]
    \end{enumerate}
\end{rem}

\begin{eg}
    \begin{enumerate}[(i)]
        \item Let 
        \begin{align*}
            E: \realn^3 \setminus \set{0} &\longrightarrow \realn^3 \\
            x &\longmapsto -\frac{x}{\norm{x}^3}
        \end{align*}
        This field is conservative, with the potential 
        \begin{align*}
            \phi: \realn^3 \setminus \set{0} &\longrightarrow \realn \\
            x &\longmapsto \rec{\norm{x}}
        \end{align*}

        \item Let 
        \begin{align*}
            E: \realn^2 \setminus \set{0} &\longrightarrow \realn^2 \\
            (x, y) &\longmapsto \left(-\frac{y}{x^2 + y^2}, \frac{x}{x^2 + y^2} \right)
        \end{align*}
        Then 
        \[
            \partial_1 E_2 = \frac{1}{x^2 + y^2} - \frac{2x^2}{(x^2 + y^2)} = \frac{y^2 - x^2}{(x^2 + y^2)^2} = \partial_2 E_1
        \]
        We can calculate the line integral of $E$ along the unit circle
        \begin{align*}
            \gamma: [0, 2\pi] &\longrightarrow \realn^2 \\
            t &\longmapsto (\cos t, \sin t)
        \end{align*}
        Then 
        \[
            E(\gamma(t)) = (-\sin t, \cos t) = \gamma'(t)
        \]
        The integral is then 
        \[
            \int_{\gamma} \innerproduct{E}{\dd{s}} = \int_0^{2\pi} \norm{(-\sin t, \cos t)}^2 \dd{t} = 2\pi \ne 0
        \]

        \item In the chapter about differential equations we looked at an exact equation in \Cref{eg:817}:
        \[
            (2x + y^2) + (2xy)y' = 0
        \]
        We can now use curve integrals to calculate the potential function more easily. For that let $x_0 = (0, 0)$.
        Then for $(\xi, \eta)$ we can define a curve connecting $x_0$ and $(\xi, \eta)$ for $t \in [0, 1]$:
        \[
            t \longmapsto (\xi t, \eta t)
        \]
        Consider the vector field
        \[
            E(x, y) = (2x + y^2, 2xy)
        \]
        Then 
        \begin{align*}
            (\xi, \eta) \longmapsto \int_{\gamma} \innerproduct{E}{\dd{s}} &= \int_0^1 \innerproduct{E(\xi t, \eta t)}{(\xi, \eta)} \dd{t} \\
            &= \int_0^1 (2\xi^2 t + \eta^2\xi t^2 + 2\xi\eta^2 t^2) \dd{t} \\
            &= \xi^2 + \eta^2\xi
        \end{align*}
    \end{enumerate}
\end{eg}

\begin{thm}
    Let $U \subset \realn^n$ be an open subset. A continuous vector field $E: U \rightarrow \realn^n$ is conservative if and only if 
    for every closed curve $\gamma: [0, 1] \rightarrow U$ the following holds
    \[
        \oint_{\gamma} \innerproduct{E}{\dd{s}} = 0
    \]
\end{thm}
\begin{proof}
    Line integrals over $E$ are path independent. Let $\gamma, \rho: [0, 1] \rightarrow U$ be paths with 
    \begin{align}
        \gamma(0) = \rho(0) && \gamma(1) = \rho(1)
    \end{align}
    Then $\gamma \inv{\rho}$ is closed, so 
    \begin{equation}
        0 = \int_{\gamma\inv{\rho}} \innerproduct{E}{\dd{s}} = \int_{\gamma} \innerproduct{E}{\dd{s}} - \int_{\rho} \innerproduct{E}{\dd{s}}
    \end{equation}
    Assume that $U$ is path continuous. Choose a fixed $x_0 \in U$ and let $g: U \rightarrow \realn$. Then 
    \begin{equation}
        g(x) = \int_{x_0}^x \innerproduct{E}{\dd{s}}
    \end{equation}
    Performing a directional derivation in direction $h \in \realn^n$ yields
    \begin{equation}
        \begin{split}
            g(x + ah) - g(x) &= \int_{x_0}^{x + ah} \innerproduct{E}{\dd{s}} - \int_{x_0}^x \innerproduct{E}{\dd{s}} \\
            &= \int_{x}^{x + ah} \innerproduct{E}{\dd{s}} \\
            &= \int_0^a \innerproduct{E(x + th)}{h} \dd{t}
        \end{split}
    \end{equation}
    Here we have chosen a linear path of integration between $x_0$ and $x$, and between $x$ and $x + ah$. In other words, we're integrating along 
    \begin{equation}
        t \longmapsto x + th
    \end{equation}
    Using the intermediate value theorem, we can find that $\exists \xi_a \in (0, a)$ such that 
    \begin{equation}
        \int_0^a \innerproduct{E(x + th)}{h} \dd{t} = \innerproduct{E(x + \xi_a h)}{h} \cdot a
    \end{equation}
    Then we have 
    \begin{equation}
        \partial_h g(x) = \limes{a}{0} \frac{g(x + ah) - g(x)}{a} = \limes{a}{0} \innerproduct{E(x + \xi_a h)}{h} = \innerproduct{E(x)}{h}
    \end{equation}
    So if $h$ is a standard basis $e_i$, then 
    \begin{equation}
        \partial_i g(x) = E_i(x)
    \end{equation}
    Thus the partial derivative of $g$ is continuous, and therefore $g$ is continuously differentiable, and thus a potential.
\end{proof}
\end{document}