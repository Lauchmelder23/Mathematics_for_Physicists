% !TeX root = ../../script.tex
\documentclass[../../script.tex]{subfiles}

\begin{document}
\section{Solution Methods}

\begin{defi}
    An ordinary differential equation (ODE) is an equation of the form 
    \[
        F(x, y, y', \cdots, y^{(n)}) = 0
    \]
    with $F: \realn^{n+2} \rightarrow \realn$. $n$ is the order of the ODE.
    Let $I$ be an open interval. A function $y: I \rightarrow \realn$ is a solution of the ODE if $y \in C^n(\realn)$ and 
    \[
        F(x, y(x), y'(x), \cdots, y^{(n)}(x)) = 0 ~~\forall x \in I
    \]
\end{defi}

\begin{eg}
    \begin{align*}
        y'' = -\frac{1}{y^2} && \text{Gravitational field} \\
        y'' = -\sin y && \text{Pendulum}
    \end{align*}
\end{eg}

\begin{rem}
    \begin{enumerate}[(i)]
        \item Often times $F$ is only defined on subsets of $\realn^{n+2}$
        \item ODEs are not simple to solve
        \item Even if we can't calculate explicit solutions, we can inspect the following properties
        \begin{itemize}
            \item Existence of solutions
            \item Uniqueness of solutions
            \item Dependency of solutions from initial conditions
            \item Sability
        \end{itemize}
    \end{enumerate}
\end{rem}

\begin{eg}
    \begin{enumerate}[(i)]
        \item Let $I$ be an open interval and $f: I \rightarrow \realn$ continuous. Then the solution of 
        \[
            y' = f(x)
        \]
        is the antiderivative of $f$. Let $x_0 \in I$, then 
        \[
            y(x) = \int_{x_0}^x f(t) \dd{t} + c ~~c \in \realn
        \]
    
        \item Consider the ODE 
        \[
            y' = y
        \]
        The functions $x \mapsto c e^x$ are solutions $\forall c \in \realn$. Are those all the solutions that exist?
        Let $y: I \rightarrow \realn$ be any solution, and consider
        \[
            u(x) = y(x)e^{-x}
        \]
        Then 
        \begin{align*}
            u'(x) &= y'(x) e^{-x} - y(x)e^{-x} \\
            &= \left(y'(x) - y(x)\right) e^{-x} = 0 ~~\forall x \in I
        \end{align*}
        So $u(x) = c$.
    \end{enumerate}
\end{eg}

\begin{defi}[Initial Value Problem]
    Let $y_0, \cdots, y_{n-1} \in \realn$ and also $F: \realn^{n+2} \rightarrow \realn$. The system of equations 
    \begin{align*}
        F(x, y, y', \cdots, y^{(n)}) = 0 && \begin{cases}
            y(0) = y_0 \\ 
            y'(0) = y_1\\
            \cdots \\ 
            y^{(n-1)}(0) = y_{n-1}
        \end{cases}
    \end{align*}
    is said to be an initial value problem (IVP).
\end{defi}

\begin{eg}
    Consider the problem 
    \begin{align*}
        y'' = -\rec{y^2} && \begin{cases}
            y(0) = y_0 \\
            y'(0) = y_1
        \end{cases}
    \end{align*}
    This describes the movement of a point mass in the gravitational field of the earth along a straight line 
    through the center of the earth with the initial position $y_0$ and the initial velocity $y_1$.
\end{eg}

\begin{eg}
    Consider the problem 
    \begin{align*}
        y' = -y^2 && y(0) = 1
    \end{align*}
    Assume $y: I \rightarrow \realn$ is a solution and $y(x) > 0 ~~\forall x \in I$. Then 
    \[
        1 = -\frac{1}{y(t)^2} ~y'(t) ~~\forall t \in I
    \]
    By integrating we get 
    \begin{align*}
        x = -\int_0^x \frac{1}{y(t)^2} y'(t) \dd{t} &\equalexpl{Substitution} -\int_1^{y(x)} \rec{y^2} \dd{y} \\
        &= \left. \rec{y} \right\vert_1^{y(x)} = \rec{y(x)} - 1 ~~\forall x \in I
    \end{align*}
    So a solution is 
    \[
        y(x) = \frac{1}{1+x}
    \]
    The biggest domain that makes sense is $(-1, \infty)$. Analogously one can approach equations with "separated variables", so of the form 
    \begin{align*}
        y' = f(y)g(x) && y(x_0) = y_0
    \end{align*}
\end{eg}

\begin{thm}[Separation of Variables]
    Let $I, J$ be open intervals, and let 
    \begin{align*}
        f: I \longrightarrow \realn && g: J \longrightarrow \realn 
    \end{align*}
    be continuous with $0 \ne f(I)$. Let $x_0 \in J, ~y_0 \in I$.
    Then there exists an open interval $I_2 \subset J$ and $x_0 \in I_2$ such that the IVP
    \begin{align*}
        y' = f(y)g(x) && y(x_0) = y_0
    \end{align*}
    has exactly one solution on $I_2$. Set 
    \[
        F(y) = \int_{y_0}^y \rec{f(t)} \dd{t}
    \]
    Then $y: I_2 \rightarrow I$ is uniquely defined by 
    \[
        F(y(x)) = \int_{x_0}^x g(t) \dd{t}
    \]
\end{thm}
\begin{proof}
    $f$ does not have any roots, thus w.l.o.g. $f > 0$.
    \begin{equation}
        F'(y) = \rec{f(y)} > 0 \implies F \text{ strictly monotonically increasing}
    \end{equation}
    Therefore there exists an inverse function $H: F(I) \rightarrow I$. According to the theorem about inverse functions, $H$ is $C^1$ and 
    \begin{equation}
        H'(z) = \rec{F'(H(z))} ~~\forall z \in F(I)
    \end{equation}
    $F(I)$ is an open interval containing the $0$. Then we have 
    \begin{equation}
        y(x) = H(G(x)) ~~x \in I_2
    \end{equation}
    where 
    \begin{equation}
        G(x) = \int_{x_0}^x g(t) \dd{t}
    \end{equation}
    Now choose $I_2$ such that $x_0 \in I_2$ and $G(I_2) \subset F(I)$. Then 
    \begin{equation}
        \begin{split}
            y'(x) &= H'(G(x)) \cdot G'(x) \\
            &= \rec{F'(H(G(x)))} \cdot G'(x) \\
            &= \rec{F'(y(x))} \cdot G'(x) \\
            &= f(y(x)) g(x)
        \end{split}
    \end{equation}
    So $y$ solves the ODE. However, if $\tilde{y}: I \rightarrow \realn$ some solution of the IVP, then $\forall x \in I_2$
    \begin{equation}
        \begin{split}
            G(x) = \int_{x_0}^x g(x) \dd{x} = \int_{x_0}^x \frac{\tilde{y}(x)}{f(\tilde{y}(x))} \dd{x} = \int_{\tilde{y}(x_0)}^{\tilde{y}(x)} \rec{f(y)} \dd{y} = F(\tilde{y}(x))
        \end{split}
    \end{equation}
    So $\tilde{y}(x) = H(G(x))$
\end{proof}

\begin{rem}
    $I_2$ is obviously not unique. We can find the biggest possible domain with 
    \[
        \bigcup_{\substack{x \in I_2 \\ I_2 \text{ open} \\ G(I_2) \subset F(I)}} I_2 = I_{2, \max}
    \]
\end{rem}

\begin{thm}
    Let $f: \realn \rightarrow \realn$ be a continuous function, $a, b, c \in \realn$ and $I$ an open interval.
    Then $y: I \rightarrow \realn$ is a solution of the ODE 
    \[
        y' = f(ax + by + c)
    \]
    if and only if $ u(x) := ax + by + c$ is a solution of 
    \[
        u' = a + bf(u)
    \]
\end{thm}
\begin{hproof}
    Consider 
    \[
        u'(x) = a + by'(x)
    \]
\end{hproof}

\begin{eg}[Euler Homogeneous ODE]
    Let $f: \realn \rightarrow \realn$ be a function and $I$ an open interval not containing the $0$.
    Then $y: I \rightarrow \realn$ is a solution of the ODE 
    \[
        y' = f(\frac{y}{x})
    \]
    if and only if 
    \[
        u(x) = \frac{y(x)}{x}
    \]
    solves the ODE
    \[
        u' = \frac{f(u) - u}{x}
    \]
\end{eg}

\begin{eg}
    Let $f: \realn \rightarrow \realn$ be continuous and $a_1, a_2, b_1, b_2, c_1, c_2 \in \realn$ such that 
    \[
        \begin{vmatrix}
            a_1 && b_1 \\
            a_2 && b_2
        \end{vmatrix} \ne 0
    \]
    Now let $\tilde{x}, \tilde{y}$ be the solutions of the equation system 
    \begin{align*}
        a_1\tilde{x} + b_1\tilde{y} + c_1 &= 0 \\
        a_2\tilde{x} + b_2\tilde{y} + c_2 &= 0
    \end{align*}
    Let $I$ be an open interval not containing the $0$. Then $y: I \rightarrow \realn$ is a solution to 
    \[
        y' = f\left(\frac{a_1x + b_1y + c_1}{a_2x + b_2y + c_2}\right)
    \]
    if and only if 
    \begin{align*}
        u: I - \tilde{x} &\longrightarrow \realn \\
        x &\longmapsto y(x + \tilde{x}) - \tilde{y}
    \end{align*}
    is a solution to 
    \[
        u' = f\left(\frac{a_1 + b_y \frac{u}{x}}{a_2 + b_2 \frac{u}{x}}\right)
    \]
\end{eg}
\begin{proof}
    Let $y: I \rightarrow \realn$ be a solution to the initial equation. Then 
    \begin{equation}
        \begin{split}
            u'(x) &= y'(x + \tilde{x}) = f\left( \frac{a_1(x + \tilde{x}) + b_1 y(x + \tilde{x}) + c_1}{a_2(x + \tilde{x}) + b_2y(x + \tilde{x}) + c_2} \right) \\
            &= f\left( \frac{a_1x + b_1u(x) + a_1\tilde{x} + b_1\tilde{y} + c_1}{a_2x + b_2u(x) + a_2\tilde{x} + b_2\tilde{y} + c_2} \right) \\
            &= f\left(\frac{a_1 + b_1 \frac{u(x)}{x}}{a_2 + b_2 \frac{u(x)}{x}}\right)
        \end{split}
    \end{equation}
    The other direction is left to the reader.
\end{proof}

\begin{defi}[Exact ODE]
    Let $D \subset \realn^2$ be open, and $p, q: D \rightarrow \realn$ continuous.
    The ODE 
    \[
        p(x, y) + q(x, y) y' = 0
    \]
    is said to be exact if there exists a $C^1$-function $H: D \rightarrow \realn$, such that
    \begin{align*}
        \partial_1 H = p && \partial_2 H = q
    \end{align*}
    Such a function is called a potential function.
\end{defi}

\begin{thm}
    Let $D \subset \realn^2$ be open, and $p, q: D \rightarrow \realn$ continuous. Let 
    \[
        p(x, y) + q(x, y)y' = 0
    \]
    be exact and $H$ a potential function. Furthermore let $I$ be an open interval and $y: I \rightarrow \realn$ a $C^1$-function such that 
    \[
        \set[x \in I]{(x, y(x))} \subset D
    \]
    Then $y$ solves the ODE if and only if $\exists c \in \realn$ such that 
    \[
        H(x, y(x)) = c
    \]
\end{thm}
\begin{proof}
    \begin{equation}
        \begin{split}
            \dv{x} H(x, y(x)) &= \partial_1 H(x, y(x)) + \partial_2 H(x, y(x)) y'(x) \\
            &= p(x, y) + q(x, y)y'(x)
        \end{split}
    \end{equation}
\end{proof}

\begin{thm}
    Let $D \subset \realn^2$ be open, and $p, q: D \rightarrow \realn$ continuously differentiable. If 
    \[
        p(x, y) + q(x, y)y' = 0
    \]
    is exact, then 
    \[
        \partial_2 p = \partial_1 q
    \]
\end{thm}
\begin{proof}
    Let $H$ be a potential $C^2$-function. Then 
    \begin{equation}
        \partial_2 p = \partial_2 \partial_1 H = \partial_1 \partial_2 H = \partial_1 q
    \end{equation}
\end{proof}

\begin{rem}
    The above condition is merely necessary! However, for "nice" $D$ it can be considered sufficient.
\end{rem}

\begin{eg}\label{eg:817}
    Consider 
    \begin{align*}
        \underbrace{(2x + y^2)}_p + \underbrace{2xyy'}_q = 0 && y(1) = 1
    \end{align*}
    Then 
    \begin{align*}
        \partial_2 p = 2y && \partial_1 q = 2y
    \end{align*}
    So $\partial_2 p = \partial_1 q$. If $H$ is a potential function, then 
    \begin{align*}
        \partial_1 H(x, y) &= p(x, y) = 2x + y^2 \\
        \implies H(x, y) &= \int p(x, y) \dd{x} = x^2 + y^2x + G(y)
    \end{align*}
    and 
    \begin{align*}
        \partial_2 H(x, y) &= q(x, y) = 2xy = 2xy + G'(y) \\
        \implies G(y) &= c
    \end{align*}
    So the potential function is 
    \[
        H(x, y) = x^2 + y^2 x
    \]
    We can insert the initial condition 
    \[
        H(1, 1) = 2
    \]
    So the solution has to fulfil 
    \[
        x^2 + y(x)^2 x = 2 ~~\forall x \in I
    \]
    and thus 
    \[
        y(x) = \pm \sqrt{\frac{2}{x} - x}
    \]
    Only the positive sign fulfils the initial conditions, so the solution is 
    \[
        y(x) = \sqrt{\frac{2}{x} - x}
    \]
    This function is defined on $(-\infty, -\sqrt{2}] \cup (0, \sqrt{2}]$, however due to the initial conditions $(0, \sqrt{2}]$ is the only useful domain.
\end{eg}

\begin{rem}
    If 
    \[ 
        p(x, y) + q(x, y)y' = 0
    \]
    is not exact one can try and find an "integrating factor", i.e. $h: D \rightarrow \realn$ such that 
    \[
        h(x, y)p(x, y) + h(x, y)q(x, y)y' = 0
    \]
    is exact. A necessary condition is 
    \[
        \left(\partial_2 h(x, y)\right)p(x, y) + h(x, y) \partial_2 p(x, y) = \left(\partial_1 h(x, y)\right) q(x, y) + h(x, y) \partial_1 q(x, y)
    \]
    This is a partial differential equation and won't be discussed further in this chapter.
\end{rem}

\begin{defi}[Ordinary Differential Equation System]
    An ordinary differential equation system (ODES) is an equation of the form 
    \[
        F(x, y, y', \cdots, y^{(n)}) = 0
    \]
    with 
    \[
        F: \realn \times \realn^L \times \realn^L \times \cdots \times \realn^L \longrightarrow \realn^m
    \]
\end{defi}

\begin{eg}
    \begin{enumerate}[(i)]
        \item Let $z = (z_1, z_2, z_3)$, then 
        \[
            z'' = -\frac{z}{\norm{z}^3} = -\rec{\norm{z}^2} \frac{z}{\norm{z}}
        \]
        is the Kepler problem.

        \item The equation 
        \begin{align*}
            b' &= \alpha_1 b - \gamma_1 br \\
            r' &= -\alpha_2 r + \gamma_2 br 
        \end{align*}
        is called the "Lotka-Volterra-Equation" and it models the population of prey and predators.
    \end{enumerate}
\end{eg}

\begin{rem}
    The ODES 
    \[
        F(x, y, y', y'', \cdots, y^{((n)}) = 0
    \]
    is equivalent to the ODES of first order 
    \begin{align*}
        F(x, y, y_1, y_2, \cdots, y_{n-1}) = 0 && \begin{cases}
            y_1 = y' \\
            y_2 = y_1' \\
            \vdots
        \end{cases}
    \end{align*}
\end{rem}
\end{document}